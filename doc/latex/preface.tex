\chapter{Preface}


The goal of {\tt pycellerator} is to provide a software tool for biological simulation that 
\begin{itemize}
\item Has all of the capabilities of the {\tt Cellerator} arrow based reaction language\cite{Cellerator};
\item Is open source;
\item Runs under Linux, Windows, and Mac OS X;
\item Is written in a multi-platform language;
\item May be used either as a stand-alone program, at the command-line, or as a library within a high-level computer language;
\item Is written in a computer language that is freely available (at no cost to users) on all platforms;
\item Does not depend on any external libraries except for freely available multi-platform libraries, so that it would not be necessary to maintain separated executable (or any executable) builds.
\end{itemize}

Python 2.7 was chosen as the programming platform because it is implemented on all three platforms, is freely available, has a large user support community, and has an extensive collect of external libraries that support it. 

The last requirement is more complicated, because it rules out a lot of useful libraries that are written in C or Fortran and wrapped in Python, and hence required complicated installations (read: recompilations) for each Operating System. It is our contention that this software should not be operating system dependent and thus the final requirement was added. 

{\tt pycellerator} was developed using standard distributions provided by the Python community (\url{http://www.python.org}). It is fully compatible with standard distributions available commercially such as Anaconda and Enthought Canopy. It can also be used in ipython notebooks. 